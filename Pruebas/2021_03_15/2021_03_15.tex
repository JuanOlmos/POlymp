\newpage
\section*{Referencias}
\textbf{Básico: }
\begin{itemize}
	\item \textit{Problema 1.} Creado 
	\item \textit{Problema 2.} UAN Clasificatorio Basico 2017.
	\item \textit{Problema 3.} UAN Clasificatorio Basico 2017.
	\item \textit{Problema 4.} UAN Clasificatorio Basico 2017.
\end{itemize}

\textbf{Avanzado: }
\begin{itemize}
	\item \textit{Problema 1.} 
	\item	\textit{Problema 2.} 
	\item	\textit{Problema 3.} 
	\item	\textit{Problema 4.}
	\item	\textit{Problema 5.} 
\end{itemize}

%------------------------------------------------------------------------------------------------------------   
%----------------------------------                        BASICO                       ---------------------------------- 
%------------------------------------------------------------------------------------------------------------ 

\newpage
\section{Nivel Básico}

\begin{center}
	\fbox{\fbox{\parbox{6in}{\centering
				\textbf{Tiempo: } 2 horas.\\
				\textbf{Procedimientos: }Cada problema debe estar resuelto por escrito en papel, en forma detallada, todos los pasos seguidos para su resolución deben estar bien explicados. Se le brindarán unas hojas grapadas, en la \textit{parte de enfrente} de cada hoja debe estar la solución de los problemas, la \textit{parte posterior} no se leerá pero las operaciones y cálculos deben hacerlos allí. \\
				\textbf{Puntaje Máximo: } 300 puntos.
				}}}
\end{center}

\begin{enumerate}
	\item \textbf{(100 puntos)}. Para año nuevo Mario se promete ahorrar en su alcancia 5 monedas el día primer día del año, 10 monedas el día segundo del año, 15 monedas el tercer día, 20 el cuarto día y así sucesivamente.
	\begin{enumerate}[label=\Alph*)]
		\item Cuántos días hay desde el primer día (contandolo) en el que mete 5 monedas hasta el día en que mete 200 monedas en la alcancía?
		
		\item Cuántas monedas habrá en su alcancia luego de que meta las 200 monedas ese día? 
		
		\item Si ahorra con monedas de 50 pesos, cuanto ahorrará en 365 dias?
		
		\item Mario quiere comprar una casa que cuesta $100.000.000$. Si continúa ahorrando de esta manera, cuántos días tienen que pasar para poder comprarla?
	\end{enumerate}
			

	\item \textbf{(50 puntos)}. Encontrar el valor de la expresión:
	\[100-98+96-94+92-90+\cdots +8-6+4-2\]
	
				
	\item \textbf{(30 puntos)}.  ¿Cuál es la suma de los diferentes divisores enteros primos de
	2016?
			
		
	\item \textbf{(60 puntos)}. Una contraseña de una tarjeta débito en el banco de Federico
	está compuesta por cuatro cifras de 0 a 9 que se pueden repetir.	Si ninguna contraseña puede comenzar con la secuencia 9, 1, 1,
	¿Cuántas contraseñas se pueden generar?
	
	\item \textbf{(60 puntos)}. La suma de 25 enteros pares consecutivos es 10, 000. ¿Cuál es
	el mayor de estos 25 enteros pares consecutivos?
	
\end{enumerate}



%------------------------------------------------------------------------------------------------------------   
%----------------------------------                        AVANZADO                       ---------------------------------- 
%------------------------------------------------------------------------------------------------------------ 

\newpage
\section{Nivel Avanzado}

\begin{center}
	\fbox{\fbox{\parbox{6in}{\centering
				\textbf{Tiempo mínimo: } 2 horas y 30 minutos.\\
				\textbf{Tiempo máximo: } 4 horas.\\		
				\textbf{Procedimientos: }Cada problema debe estar resuelto por escrito, en forma detallada, todos los pasos seguidos para su resolución deben estar bien explicados. Se le brindarán unas hojas grapadas, en la \textit{parte de enfrente} de cada hoja debe estar la solución de los problemas, la \textit{parte posterior} no se leerá pero las operaciones y cálculos deben hacerlos allí. \\
				\textbf{Puntaje: }Cada problema vale 50 puntos, son 5, para un total de 250 puntos.
	}}}
\end{center}


\begin{enumerate}
	\item \textbf{(50 puntos)}. 
	

	\item \textbf{(50 puntos)}. 
	

	\item \textbf{(50 puntos)}.  

	\item \textbf{(50 puntos)}. 

	\item \textbf{(50 puntos)}. 

\end{enumerate}