\chapter{Combinatoria}\label{sol:chapter:combinatoria}



\saleAEjercicios{combinatoria:ContandoListasDeNumeros}
\section{Ejercicios \ref{ejercicios:part:combinatoria:chp:contando_listas_de_numeros}}
\entraASolution{solution:combinatoria:ContandoListasDeNumeros}
\begin{enumerate}
	\item Restanto 35 a todos: $1,2,3,\cdots, 58$. Entonces hay 58.
	\item Dividiendo en 2 a todos: $2,3,4,\cdots, 65$. Entonces hay 64.
	\item Restando 2 a todos: $4,8,12,\cdots, 84$. Dividiendo en 4: $1,2,3,\cdots, 21$. Entonces hay 21.
	\item Dividiendo en 3 a todos: $49,48,47,\cdots, 13$. Entonces hay 37.
	\item Restando 2 a todos: $-35, -30, \cdots, 55$. Dividiendo en 5 $-7,-6, \cdots, 11$. Entonces hay 19. 
	\item Sumando $0.5$ a todos: $3,6,9,\cdots, 84$ dividiendo en 3, $1,2,3,\cdots, 28$. Entonces hay 28.
	\item $7=7\times 1,14=7\times 2,\cdots, 149=7\times 21$. Entonces hay 21.
	\item $126=42\times 3, \cdots, 9996=42\times 238$. Entonces hay 236.
	
	\item $243=3^3,\cdots 6561=3^8$. Entonces hay 6.
	

	\item La diferencia entre uno y otro es $\frac{2}{3}$. Escribiendo la lista de otra forma $\frac{11}{3}, \frac{13}{3}, \frac{15}{3}, \cdots, \frac{81}{3}$.Si multiplicamos todo por 3, $11,13,15, \cdots, 81=10,12,14,\cdots, 80=5,6,7,\cdots40$ . Entonces hay 36.

	
	\item $1024=32^2,\cdots, 9801=99^2$. Entonces hay $99-31=68$.
	
	\item	$\{1,2,3,4\}=24, \{2,3,4,5\}=120, \cdots \{16,17,18,19\}=93024$. Entonces hay 16.

\end{enumerate}

\chapter{Geometr\'ia}\label{sol:chapter:geometria}


\chapter{Teor\'ia de N\'umeros}\label{sol:chapter:teoria_de_numeros}


\chapter{\'Algebra}\label{sol:chapter:algebra}